\documentclass[11pt, a4paper]{article}

% --- 1. PREAMBLE ---
% --- Geometry & Font ---
\usepackage[a4paper, top=2.5cm, bottom=2.5cm, left=2.5cm, right=2.5cm]{geometry}
\usepackage[T1]{fontenc}
\usepackage[english]{babel}
\usepackage{lmodern}
\renewcommand{\familydefault}{\sfdefault}

% --- Standard Packages ---
\usepackage{amsmath}
\usepackage{graphicx}
\usepackage{booktabs}
\usepackage{caption}
\usepackage{verbatim}
\usepackage{textcomp}

% --- Section Styling ---
\usepackage{sectsty}
\sectionfont{\sffamily\Large}
\subsectionfont{\sffamily\large}
\subsubsectionfont{\sffamily\normalsize}

% --- Hyperlinks ---
\usepackage{hyperref}
\hypersetup{
    colorlinks=true,
    linkcolor=blue,
    filecolor=magenta,
    urlcolor=cyan,
    pdftitle={GEMM Optimization Report},
    pdfauthor={Your Team},
}

% --- 2. TITLE SECTION ---
\title{
    \sffamily\bfseries\huge
    Lab: Accelerating Code: From Python Baseline to Architecture-Aware Optimization \\
    \vspace{0.5\baselineskip}
    \large Dense Matrix-Matrix Multiplication (GEMM) Challenge
}

\author{
    \sffamily
    Team Name: [Your Team Name Here] \\
    Entry Numbers: [mcsXXXXX, mcsYYYYY]
}

\date{\today}

% --- 3. DOCUMENT BODY ---
\begin{document}

\maketitle

\begin{abstract}
\noindent
This report details the iterative process of optimizing a Dense Matrix-Matrix Multiplication (GEMM) algorithm, starting from a Python baseline. We describe the implementation of several architecture-aware techniques, including OpenMP for multicore parallelism, cache-friendly blocking, and SIMD vectorization using AVX intrinsics. We present a comprehensive performance analysis, measuring GFLOPS, speedup, and scalability against thread count and problem size. Our final optimized C++ implementation achieves a significant speedup of [X.XX]x over the baseline, demonstrating the performance impact of architecture-level design.
\end{abstract}

\tableofcontents
\newpage

% --- Section 1 ---
\section{Problem Description}

The objective of this assignment is to accelerate a Dense Matrix-Matrix Multiplication (GEMM) operation, $C = A \times B$. The baseline implementation is provided in Python, using \texttt{numpy} and the \texttt{multiprocessing} library. Our goal is to reimplement this in C++ and apply a series of architecture-aware optimizations to maximize performance, measured in GFLOPS and speedup.

This report documents our optimizations for input matrix sizes
\[
N \in \{512, 1024, 1536, 2048\}.
\]

% --- Section 2 ---
\section{Baseline Implementation}

The baseline consists of the \texttt{gemm\_baseline.py} script. It uses the Python \texttt{multiprocessing} library to distribute blocks of matrix A to worker processes, where each worker computes its portion using \texttt{np.dot()}. Although \texttt{numpy} is optimized, the Python overhead is significant.

Our environment:

\begin{itemize}
    \item \textbf{CPU Model:} [Your CPU Model]
    \item \textbf{Cores/Threads:} [e.g., 16 cores / 32 threads]
    \item \textbf{Caches:} [e.g., L1/L2/L3 = 32K/1024K/36608K]
    \item \textbf{OS:} [e.g., Ubuntu 20.04.5 LTS]
    \item \textbf{Compiler:} g++ 11.3.0
\end{itemize}

% --- Section 3 ---
\section{Optimizations Implemented}

We implemented the following optimizations.

\subsection{Multithreading with OpenMP}

A major bug was fixed—OpenMP was incorrectly placed inside inner loops. The final correct placement:

\begin{verbatim}
// Correct OpenMP structure
#pragma omp parallel for schedule(static)
for (int i0 = 0; i0 < N; i0 += MC) {
    for (int k0 = 0; k0 < N; k0 += KC) {
        for (int j0 = 0; j0 < N; j0 += NC) {
            // ...
        }
    }
}
\end{verbatim}

\subsection{Cache-Friendly Blocking}

To reduce cache misses, we divided matrices into blocks:

- A: \(MC \times KC\)
- B: \(KC \times NC\)
- C: \(MC \times NC\)

We chose: \(MC = 128\), \(KC = 256\), \(NC = 256\).

\begin{figure}[htbp]
  \centering
  \framebox{\parbox{0.8\textwidth}{
    \centering
    \vspace{4cm}
    \textbf{Image Placeholder: Cache Blocking Diagram} \\
    \small\textit{Block-level diagram showing i0, k0, j0 tiling.}
    \vspace{4cm}
  }}
  \caption{Our cache-blocking strategy.}
  \label{fig:blocking}
\end{figure}

\subsection{SIMD Vectorization}

We implemented an AVX2/AVX-512 microkernel to compute the innermost block fully in vector registers.

\subsection{Other Optimizations}

\begin{itemize}
    \item Pre-transposition of matrix B for sequential memory access.
    \item 64-byte aligned memory for SIMD.
    \item Compiler flags: \texttt{-O3 -march=native -fopenmp}.
\end{itemize}

% --- Section 4 ---
\section{Experimental Methodology}

We used automated scripts:

\begin{enumerate}
    \item Warmup run
    \item 10 measured runs per configuration
    \item Logged to \texttt{gemm\_results.csv}
\end{enumerate}

% --- Section 5 ---
\section{Performance Results}

\begin{table}[htbp]
\centering
\caption{Key Performance Metrics (Avg. of 10 Runs)}
\label{tab:results}
\begin{tabular}{lrrrr}
\toprule
\textbf{Metric} & \textbf{512} & \textbf{1024} & \textbf{1536} & \textbf{2048} \\
\midrule
Baseline Time (s) & [ ] & [ ] & [ ] & [ ] \\
Optimized Time (s, T=16) & [ ] & [ ] & [ ] & [ ] \\
Optimized GFLOPS & [ ] & [ ] & [ ] & [ ] \\
Max Speedup & [ ] & [ ] & [ ] & [ ] \\
\bottomrule
\end{tabular}
\end{table}

\begin{figure}[htbp]
  \centering
  \framebox{\parbox{0.8\textwidth}{
    \centering
    \vspace{5cm}
    \textbf{Image Placeholder: GFLOPS vs Threads Plot} \\
    \vspace{5cm}
  }}
  \caption{Performance (GFLOPS) vs. Thread Count.}
\end{figure}

\begin{figure}[htbp]
  \centering
  \framebox{\parbox{0.8\textwidth}{
    \centering
    \vspace{5cm}
    \textbf{Image Placeholder: Speedup vs Threads Plot} \\
    \vspace{5cm}
  }}
  \caption{Speedup vs. Thread Count.}
\end{figure}

\end{document}
